% !TeX root = ./main.tex

\begin{savenotes}
\begin{table}[htbp]
    \centering
    \caption{インシデントタイムライン}
    \label{tab:incident_timeline}
    \begin{tabular}{|l|p{8cm}|}
    \hline
    2017年3月6日 & Apache Struts 2 に関する脆弱性 S2-045 が Wiki\footnote{
        S2-045 - Apache Struts 2 Wiki - Apache Software Foundation
        : \url{https://cwiki.apache.org/confluence/pages/viewpage.action?pageId=68717772}
    }
    に投稿される(3/2にV1が出ているが,バージョン 2.3.32 について言及されておらず不完全と判断した) \\ \hline
    2017年3月7日 & Github で攻撃コード公開\footnote{
        xsscx氏による一例: \url{https://github.com/xsscx/cve-2017-5638}
    } \\ \hline
    2017年3月9日 & IPAが管理する,Apache Struts2の脆弱性対策情報一覧ページ
    \footnote{
        Apache Struts2 の脆弱性対策情報一覧
        : \url{https://www.ipa.go.jp/security/announce/struts2_list.html}
    }
    に S2-045 が追加される \\ \hline
    2017年3月10日 & ぴあ社が S2-045を社内で認識する \\ \hline
    2017年3月7日~15日(後の検証で判明) & ぴあ社が運営するECコンテンツ「B.LEAGUE チケットサイト,及びファンクラブ受付サイト」が何者かにサイバー攻撃を受ける.(IPAの発表によりApache Struts2の脆弱性をぴあ社内で認識していたが,webサーバ上にクレジット情報は保存されてないと認識していた) \\ \hline
    2017年3月17日 & SNSサイト「Twitter」にて,クレジットカードの不正利用に関する書き込みが続出.クレジットカード会社より,不正利用の形跡があると報告を受ける. \\ \hline
    2017年3月25日 & ぴあ社は,当該ウェブサイト上でのクレジットカード決済機能を停止.同時に第三者調査機関に調査を依頼する. \\ \hline
    2017年4月25日 & ウェブサイト上にて,不正アクセス被害の報告と謝罪を実施.予想される被害総数は15万4,599件(うちカード関係の情報は3万2,187件)と発表. \\ \hline
    2017年4月28日 & 同社CCOを委員長とする再発防止委員会を組成.再発防止に向けて取り組む. \\ \hline
    2017年5月18日 & 第三者調査機関の報告を受けて,ウェブサイト上で続報を発表.既に公表されていたカード情報3万2,187件とは別に,新たに6,508件の漏えいの可能性があることを公開. \\ \hline
    \end{tabular}
\end{table}
\end{savenotes}
